\newpage
\section*{Actividades y metodología}

\textbf{Engine}

Se implementará un engine de ajedrez con heurísticas y mejoras clásicas, usando una evaluación con NNUEs (cuantizadas y utilizando SIMD). La implementación será negamax con poda alfa-beta, incluyendo las heurísticas: ordenamiento de movimientos (MVVA, killer/history), búsqueda quiescente, null-move y tabla de transposiciones. \\

\textbf{Entrenamiento}

Para entrenar los modelos, se probarán 2 técnicas: \\

``Estándar'': Se toman posiciones aleatorias del dataset de partidas y se utiliza Stockfish (oráculo) a profundidad fija para obtener una evaluación, generando un nuevo dataset. Luego se entrena el modelo usando estos puntajes como target. Esto es lo mismo que hace Stockfish y debería ser lo mejor (es el estado del arte).

``$PQR$'': Esta técnica, inspirada en \cite{dlchess:2014}, no utiliza ningún oráculo. Se genera un nuevo dataset de triplas $(P,Q,R)$. Se toma una posición $P$ aleatoria en una partida. Luego se toma la posición observada como $Q$ (es decir, la posición siguiente en la partida, la que se jugó, $P \rightarrow Q$). Finalmente, se toma una posición $R$ aleatoria tal que $P \rightarrow R$ y $R \neq Q$. Suponiendo que $f$ es el modelo, la premisa de esta técnica es que los jugadores eligen movimientos que son buenos para ellos, pero malos para el otro, entonces $f(P)=-f(Q)$. Por la misma razón, ir de $P$ a $R$ (es decir, no $Q$) una posición aleatoria, se espera que $f(R) > f(Q)$, porque el movimiento aleatorio es mejor para el jugador siguiente y peor para el que hizo el movimiento. Se utiliza una función de pérdida con esas inecuaciones. \\


\textbf{Dataset}

Se utilizará el dataset abierto de Lichess \cite{lichessdb} (CC0) que cuenta con 5.5 miles de millones de partidas públicas jugadas en el sitio, equivalentes a 1.71TB de PGNs\footnote[1]{Portable Game Notation: un formato para grabar los movimientos de una partida y metadatos.} comprimidos. En el dataset hay más de 200 miles de millones de posiciones, pero dado la cantidad de partidas, voy a considerar solo una posición por partida, para mejorar la diversidad. Además, se utilizará el dataset de puzzles que también ofrece Lichess para evaluar el desempeño.

