\section*{Actividades y metodología}

\textbf{Engine}

Se implementa un engine de ajedrez con heurísticas y mejoras clásicas, usando una evaluación con NNUEs (cuantizadas y en SIMD). Se utilizará para medir la performance de los modelos: jugar partidas entre ellos (elo) y resolver puzzles.
La implementación será negamax, con poda alfa-beta incluyendo las heurísticas: ordenamiento de movimientos (MVVA, killer/history), búsqueda quiescente, null-move y tabla de transposiciones. \\

\textbf{Dataset}

Se utilizará el dataset de Lichess, que cuenta con 5.5B de partidas públicas jugadas en Lichess, equivalentes a 1.71TB de PGNs comprimidos. En el dataset hay más de 200B de posiciones, pero dado la cantidad de partidas, voy a considerar solo una posición por partida, para mejorar la diversidad. Además se utilizará el dataset de puzzles de Lichess para evaluar la performance. \\

\textbf{Entrenamiento}

Para entrenar los modelos, se prueban 2 alternativas:

``Eval'': Se toman posiciones aleatorias del dataset de partidas y se utiliza Stockfish a profundidad fija como oráculo para obtener una evaluación, generando un nuevo dataset. Luego se entrena el modelo usando estos puntajes como target. Esto es lo mismo que hace Stockfish y debería ser lo mejor.


\cite{dlchess:2014}

``$PQR$'': No se usa ningún oráculo, se genera un nuevo dataset de triplas $(P,Q,R)$. Se toma una posición $P$ aleatoria en una partida. Luego se toma la posición observada como $Q$ (es decir la posición siguiente en la partida, la que se jugó, $P \rightarrow Q$). Finalmente se toma una posición $R$ aleatoria tal que $P \rightarrow R$ y $R \neq Q$. Suponiendo que $f$ es el modelo, la premisa de esta técnica es que los jugadores eligen movimientos que son buenos para ellos, pero malos para los otros, entonces $f(P)=-f(Q)$. Por la misma razón, ir de $P$ a $R$ (es decir no $Q$) una posición aleatoria, se espera que $f(R) > f(Q)$, porque el movimiento aleatorio es mejor para el jugador siguiente y peor para el que hizo el movimiento. Se utiliza una función de pérdida con esas inecuaciones.
