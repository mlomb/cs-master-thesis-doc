
\subsection{Mobility}

\textbf{Motivation.} Mobility in chess is a measure of the available moves a player can make in a given position. The idea is that if a player has more available moves, the position is stronger. In \cite{slater:1950} it was shown that there is a strong correlation between a player's mobility and the number of games won. This metric has been used extensively in hand crafted evaluations, and I propose to include this information as features for the neural network.

There are a couple of ways to go about encoding mobility:

\begin{itemize}
\item \textbf{Bitsets (per piece type):} the amount of features changed each turn may negate any gains.

\begin{figure}[h]
\centering

\begin{tabular}{ccccc}

\raisebox{-7ex}{\chessboard[
    setfen=r5k1/1b1p1ppp/p7/1p1Q4/2p1r3/PP4Pq/BBP2b1P/R4R1K w - - 0 20,
    tinyboard,
    showmover=false,
]}
&

\raisebox{-7ex}{\chessboard[
    tinyboard,
    showmover=false,
    setwhite={ba2,bb2},
    pgfstyle=color,
    opacity=0.8,
    color=blue,
    markfield={b1,c1,c3,d4,e5,f6,g7}
]}

&

\raisebox{-7ex}{\chessboard[
    tinyboard,
    showmover=false,
    addblack={Bb7,Bf2},
    pgfstyle=color,
    opacity=0.8,
    color=blue,
    markfield={c8,c6,d5,a7,b6,c5,d4,e3,e1,g1,g3}
]}

&

\raisebox{-7ex}{\chessboard[
    tinyboard,
    showmover=false,
    setwhite={qd5},
    pgfstyle=color,
    opacity=0.8,
    color=blue,
    markfield={d6,d7,e6,f7,e5,f5,g5,h5,e4,d4,d3,d2,d1,c4,c5,b5,c6,b7}
]}

& $\hdots$

\\

Board &
\makecell{\white White\\\symbishop\ Bishop} &
\makecell{\black Black\\\symbishop\ Bishop} &
\makecell{\white White\\\symqueen\ Queen}

\end{tabular}
\end{figure}


\item \textbf{Counts (per piece type):}
\end{itemize}
