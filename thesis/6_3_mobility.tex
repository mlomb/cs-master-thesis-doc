
\subsection{Mobility}

\textbf{Motivation.} Mobility in chess is a measure of the available moves a player can make in a given position. The idea is that if a player has more available moves, the position is stronger. In \cite{slater:1950} it was shown that there is a strong correlation between a player's mobility and the number of games won. This metric has been used extensively in hand crafted evaluations, and I propose to include this information as features for the neural network. \\

\textbf{Experiment.} There are two ways to go about encoding mobility:

\begin{itemize}
\item \textbf{Bitsets (per piece type):} Provide the exact squares each piece type can move to. The number of features would be $64 * 6 * 2 = 768$. The problem with this approach is not the amount of features, but the number of updates to the accumulator per move is very high, which slows down the search.

\begin{figure}[h]
\centering

\begin{tabular}{ccccc}

\raisebox{-7ex}{\chessboard[
    setfen=r5k1/1b1p1ppp/p7/1p1Q4/2p1r3/PP4Pq/BBP2b1P/R4R1K w - - 0 20,
    tinyboard,
    showmover=false,
]}
&

\raisebox{-7ex}{\chessboard[
    tinyboard,
    showmover=false,
    setwhite={ba2,bb2},
    pgfstyle=color,
    opacity=0.8,
    color=blue,
    markfield={b1,c1,c3,d4,e5,f6,g7}
]}

&

\raisebox{-7ex}{\chessboard[
    tinyboard,
    showmover=false,
    addblack={Bb7,Bf2},
    pgfstyle=color,
    opacity=0.8,
    color=blue,
    markfield={c8,c6,d5,a7,b6,c5,d4,e3,e1,g1,g3}
]}

&

\raisebox{-7ex}{\chessboard[
    tinyboard,
    showmover=false,
    setwhite={qd5},
    pgfstyle=color,
    opacity=0.8,
    color=blue,
    markfield={d6,d7,e6,f7,e5,f5,g5,h5,e4,d4,d3,d2,d1,c4,c5,b5,c6,b7}
]}

& $\hdots$

\\

Board &
\makecell{\white White\\\symbishop\ Bishop} &
\makecell{\black Black\\\symbishop\ Bishop} &
\makecell{\white White\\\symqueen\ Queen}

\end{tabular}
\end{figure}


\item \textbf{Counts (per piece type):} Use the number of available moves per piece type as features. This means having a feature for each possible count value, with a lot less of row updates. To find which values to include as features I computed the total mobility for each piece role in 2 billion boards, shown in figure \ref{fig:mobility}. From the data we can extract the range of values to use as features:

\begin{table}
\centering
\begin{tabular}{c|c|c}
\toprule
\textbf{Piece role} & \textbf{Min} & \textbf{Max} \\
\midrule
\sympawn\ Pawn & 0 & 8+ \\
\symknight\ Knight & 0 & 15+ \\
\symbishop\ Bishop & 0 & 16+ \\
\symrook\ Rook & 0 & 25+ \\
\symqueen\ Queen & 0 & 25+ \\
\symking\ King & 0 & 8 \\
\bottomrule
\end{tabular}
\end{table}

\begin{figure}[H]
\centering
\makebox[\textwidth]{\includegraphics[width=\textwidth]{./dynamic/output/mobility.pdf}}
\caption{Total mobility values for each piece on the board. Computed using 2 billion boards. The value 0 for the \symknight\ Knight, \symbishop\ Bishop, \symrook\ Rook and \symqueen\ Queen has been excluded from the plot, as it is very common.}
\label{fig:mobility}
\end{figure}
\end{itemize}

Each approach was implemented as a block:

\begin{table}[H]
\caption{Mobility feature blocks}
\label{tab:mobility_blocks}
\centering

\begin{tabular}{ccc}
\toprule
\bf Block name & \bf Definition & \bf Number of features \\ 
\toprule
MB & \makecell{$\featureset{Square}_{P} \times (\featureset{Role}_{Q} \times \featureset{Color}_{Q})$ \\for every square $P$ and piece $Q$\\ such that $Q$ can move to $P$} & 768 \\
\toprule
MC & \makecell{$M(R)_{R,C} \times (\featureset{Role}_{R} \times \featureset{Color}_{C})$\\for every role $R$ and color $C$\\where $M(R)=\{0\hdots\}$ are the mobility values for the role R} & 206 \\
\bottomrule
\end{tabular}
\end{table}

decir los fs


asdasdasd \\

\textbf{Results.} asdadasdasd
