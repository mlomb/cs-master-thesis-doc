\thispagestyle{plain}
\begin{center}
\large
\textbf{Abstract}
\end{center}

\begin{center}
\parbox{15cm}{
Historically, chess engines have used highly complex functions to evaluate chess positions. Recently, efficiently updatable neural networks (NNUE) have displaced these functions that do not need human knowledge. The input of these networks are called feature sets and they take advantage of the order in which positions are evaluated in a depth-first search to save computation. \\

In this thesis, I develop a classical chess engine, where the evaluation function is replaced by a NNUE network. The main goal of this thesis is to test novel feature sets that can improve performance. Additionally, a way of training the networks is tested using a method proposed years ago but with a higher volume and quality of data available in the post-NNUE era.
}
\end{center}

\vspace{1cm}

\begin{center}
\large
\textbf{Abstract (Spanish)}
\end{center}

\begin{center}
\parbox{15cm}{
Históricamente, los motores de ajedrez han utilizado funciones altamente complejas para evaluar posiciones de ajedrez. Recientemente las redes neuronales eficientemente actualizables (NNUE) han desplazado a estas funciones sin necesidad de utilizar conocimiento humano. El input de estas redes se las denomina feature sets y se aprovechan del orden en que se evalúan las posiciones en una búsqueda depth-first para ahorrar cómputo. \\

En esta tesis realizo un motor de ajedrez clásico, en donde la función de evaluación es reemplazada por una red NNUE. Esta tesis busca probar novedosos feature sets que puedan mejorar el rendimiento. Adicionalmente, se prueba una manera de entrenar las redes utilizando un método propuesto hace años pero con un volumen y calidad de datos superiores disponibles en la era post-NNUE.
}
\end{center}

\clearpage
