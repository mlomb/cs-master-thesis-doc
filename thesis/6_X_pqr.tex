\subsection{PQR}

\textbf{Motivation.} During the initial research for a thesis subject, I came across \cite{dlchess:2014} which seemed an interesting approach to train a neural network to evaluate positions. Since it was released in 2014, it predates the NNUE era and the training data was suboptimal (Lichess with human moves). So I decided to try to replicate the idea using modern datasets and better moves. The \enquote{PQR} method itself was explained in detail in the previous chapter. \\

\textbf{Experiment.} I will train the canonical \featureset{All} feature set with this method in two ways: from scratch and from a checkpoint trained with the evaluations method.

? why...


\textbf{Results.}

PQR. no va a ser tan bueno pero fun shit

La idea es separarlo en dos:

- probar entrenando de cero
- probar entrenando desde un checkpoint ya entrenado con la otra tecnica
