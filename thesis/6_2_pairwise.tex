\subsection{Pairwise axes}

\textbf{Motivation.} Imagine that in a file there are three pieces: an enemy rook, a pawn and the king. There are many possible configurations for these pieces on the file. Of course that the influence in score is very related with the position of pieces everywhere else, but I believe that to understand a single file, the distance between the pieces are less important than the order between them: $\sympawn\symking\symrook, \sympawn\symrook\symking, \symking\sympawn\symrook, \symking\symrook\sympawn, \symrook\sympawn\symking, \symrook\symking\sympawn$.



Pepito \depiction{PH} asdo \depiction{PV} asd\depiction{PD1} asd\depiction{PD2} asd

[explicar...]

3 runs
HV + PH
HV + PV
HV + PH + PV


\begin{figure}[H]
\centering

\begin{tabular}{cc}

\raisebox{-7ex}{\chessboard[
    setfen=2r4k/p5p1/Kpqp3p/8/1PP2Q2/P2P1RP1/8/8 b - - 12 45,
    showmover=false,
    opacity=0.6,
    %
    % TEMPLATE HORIZONTAL
    %color=red,
    %shortenend=1.88ex,shortenstart=1.88ex, % espacio
    %padding=-1ex,
    %markstyle=topborder,
    %linewidth=0.4ex,
    %markregion=d6-d3,
    %linewidth=1.6ex,
    %pgfstyle=circle,
    %markfields={d6,d3},
    %
    color=red,
    shortenend=1.88ex,shortenstart=1.88ex, % espacio
    padding=-1ex,
    markstyle=topborder,
    linewidth=0.4ex,
    markregion=a3-d3,
    linewidth=1.6ex,
    pgfstyle=circle,
    markfields={a3,d3},
    %
    color=red,
    shortenend=1.88ex,shortenstart=1.88ex, % espacio
    padding=-1ex,
    markstyle=topborder,
    linewidth=0.4ex,
    markregion=f3-g3,
    linewidth=1.6ex,
    pgfstyle=circle,
    markfields={f3,g3},
    %
    color=blue,
    shortenend=1.88ex,shortenstart=1.88ex, % espacio
    padding=-1ex,
    markstyle=topborder,
    linewidth=0.4ex,
    markregion=d3-f3,
    linewidth=1.6ex,
    pgfstyle=circle,
    markfields={d3,f3},
    %
    color=red,
    shortenend=1.88ex,shortenstart=1.88ex, % espacio
    padding=-1ex,
    markstyle=topborder,
    linewidth=0.4ex,
    markregion=b4-c4,
    linewidth=1.6ex,
    pgfstyle=circle,
    markfields={b4,c4},
    %
    color=blue,
    shortenend=1.88ex,shortenstart=1.88ex, % espacio
    padding=-1ex,
    markstyle=topborder,
    linewidth=0.4ex,
    markregion=c4-f4,
    linewidth=1.6ex,
    pgfstyle=circle,
    markfields={c4,f4},
    %
    color=red,
    shortenend=1.88ex,shortenstart=1.88ex, % espacio
    padding=-1ex,
    markstyle=topborder,
    linewidth=0.4ex,
    markregion=a6-b6,
    linewidth=1.6ex,
    pgfstyle=circle,
    markfields={a6,b6},
    %
    color=red,
    shortenend=1.88ex,shortenstart=1.88ex, % espacio
    padding=-1ex,
    markstyle=topborder,
    linewidth=0.4ex,
    markregion=c6-d6,
    linewidth=1.6ex,
    pgfstyle=circle,
    markfields={c6,d6},
    %
    color=blue,
    shortenend=1.88ex,shortenstart=1.88ex, % espacio
    padding=-1ex,
    markstyle=topborder,
    linewidth=0.4ex,
    markregion=b6-c6,
    linewidth=1.6ex,
    pgfstyle=circle,
    markfields={b6,c6},
    %
    color=blue,
    shortenend=1.88ex,shortenstart=1.88ex, % espacio
    padding=-1ex,
    markstyle=topborder,
    linewidth=0.4ex,
    markregion=d6-h6,
    linewidth=1.6ex,
    pgfstyle=circle,
    markfields={d6,h6},
    %
    color=red,
    shortenend=1.88ex,shortenstart=1.88ex, % espacio
    padding=-1ex,
    markstyle=topborder,
    linewidth=0.4ex,
    markregion=a7-g7,
    linewidth=1.6ex,
    pgfstyle=circle,
    markfields={a7,g7},
    %
    color=blue,
    shortenend=1.88ex,shortenstart=1.88ex, % espacio
    padding=-1ex,
    markstyle=topborder,
    linewidth=0.4ex,
    markregion=c8-h8,
    linewidth=1.6ex,
    pgfstyle=circle,
    markfields={c8,h8},
]}

&

\raisebox{-7ex}{\chessboard[
    setfen=2r4k/p5p1/Kpqp3p/8/1PP2Q2/P2P1RP1/8/8 b - - 12 45,
    showmover=false,
    opacity=0.6,
    %
    % TEMPLATE VERTICAL
    %color=red,
    %shortenend=1.88ex,shortenstart=1.88ex, % espacio
    %padding=-1ex,
    %markstyle=leftborder,
    %linewidth=0.4ex,
    %markregion=d6-d3,
    %linewidth=1.6ex,
    %pgfstyle=circle,
    %markfields={d6,d3},
    %
    color=red,
    shortenend=1.88ex,shortenstart=1.88ex, % espacio
    padding=-1ex,
    markstyle=leftborder,
    linewidth=0.4ex,
    markregion=a3-a6,
    linewidth=1.6ex,
    pgfstyle=circle,
    markfields={a3,a6},
    %
    color=blue,
    shortenend=1.88ex,shortenstart=1.88ex, % espacio
    padding=-1ex,
    markstyle=leftborder,
    linewidth=0.4ex,
    markregion=a6-a7,
    linewidth=1.6ex,
    pgfstyle=circle,
    markfields={a6,a7},
    %
    color=red,
    shortenend=1.88ex,shortenstart=1.88ex, % espacio
    padding=-1ex,
    markstyle=leftborder,
    linewidth=0.4ex,
    markregion=b4-b6,
    linewidth=1.6ex,
    pgfstyle=circle,
    markfields={b4,b6},
    %
    color=red,
    shortenend=1.88ex,shortenstart=1.88ex, % espacio
    padding=-1ex,
    markstyle=leftborder,
    linewidth=0.4ex,
    markregion=c4-c6,
    linewidth=1.6ex,
    pgfstyle=circle,
    markfields={c4,c6},
    %
    color=blue,
    shortenend=1.88ex,shortenstart=1.88ex, % espacio
    padding=-1ex,
    markstyle=leftborder,
    linewidth=0.4ex,
    markregion=c6-c8,
    linewidth=1.6ex,
    pgfstyle=circle,
    markfields={c6,c8},
    %
    color=red,
    shortenend=1.88ex,shortenstart=1.88ex, % espacio
    padding=-1ex,
    markstyle=leftborder,
    linewidth=0.4ex,
    markregion=d3-d6,
    linewidth=1.6ex,
    pgfstyle=circle,
    markfields={d3,d6},
    %
    color=red,
    shortenend=1.88ex,shortenstart=1.88ex, % espacio
    padding=-1ex,
    markstyle=leftborder,
    linewidth=0.4ex,
    markregion=f3-f4,
    linewidth=1.6ex,
    pgfstyle=circle,
    markfields={f3,f4},
    %
    color=red,
    shortenend=1.88ex,shortenstart=1.88ex, % espacio
    padding=-1ex,
    markstyle=leftborder,
    linewidth=0.4ex,
    markregion=g3-g7,
    linewidth=1.6ex,
    pgfstyle=circle,
    markfields={g3,g7},
    %
    color=red,
    shortenend=1.88ex,shortenstart=1.88ex, % espacio
    padding=-1ex,
    markstyle=leftborder,
    linewidth=0.4ex,
    markregion=h6-h8,
    linewidth=1.6ex,
    pgfstyle=circle,
    markfields={h6,h8},
]}


\\

\makecell{\depiction{PH} Pairwise horizontal (\featureset{PH})} &
\makecell{\depiction{PV} Pairwise vertical (\featureset{PV})}

\end{tabular}
\end{figure}


el cosito de los pares \\

los resultados no son buenos...

I did not bother implementing diagonal pairs: \depiction{PD1} and \depiction{PD2}.
