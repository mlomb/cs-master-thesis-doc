\section{Final words}
\subsection{Conclusions}

maybe implementing a custom engine was not a good idea. bugs and stuff

feature sets: havent changed much in a long time. thats why it is so hard find anything better

feature engineering slow with this kind of task, iteration times are extremely slow. This is why fishnet exists.

% prunning feature sets? quizas no ayuda, solo para training, no se. ver bien

% PIECE INDEPENDENT FEATURES SETS ARE TOO GOOD

% lo ideal de un feature set son patrones que NO se den en simultáneo? asi podes aprender mas usando menos neuronas

\subsection{Future work}

Training NNUEs is a daunting task, and there are lots of variables that affect dramatically the performance of the networks. Many decisions were made in this work to reduce the scope of the project, so naturally many variables were left unexplored.

The following are some key points that could be explored in a future work: \\

% deduplication de posiciones (al computar el score de Stockfish)
\textbf{Dataset:}
A great deal of effort is put into good training data: the source, the filtering, ........ 
\\

\textbf{Alternative to PQR:} Instead of the loss function used to train PQR, the triplet loss function could be tried, where the anchor is the P position, the positive is the observed position (Q) and the negative is the random position (R). I don't expect this to improve the that much, but it is worth trying. \\

\textbf{Network architecture:} The architecture of NNUE-like networks has gone through multiple iterations since its inception. This work focused on the first and most basic iteration of it. Maybe it is worth exploring more complex architectures with a fixed feature set rather than a fixed architecture with a variable feature set.
Almost certainly try lower values of L2, which may bring better results. \\

\textbf{Feature sets:} more fs, prunning de features al pepe para evitar updates
