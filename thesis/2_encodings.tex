\newcommand{\white}{\fullmoon}
\newcommand{\black}{\newmoon}

\newcommand{\bigtimes}{\mathop{\raisebox{-0.5ex}{\scalebox{2}{$\times$}}}}

% https://texdoc.org/serve/chessboard/0
\newcounter{pieceindex}
\newcommand{\pieceBoard}{
    \newcount\pieceindex
    \setcounter{pieceindex}{0}
    \raisebox{-7ex}{
        \centering
        \chessboard[
            tinyboard,
            showmover=false,
            margin=false,
            padding=false,
            hlabel=false,
            vlabel=false,
            pgfstyle={text},
            %text=\fontsize{1.2ex}{1.2ex}\bfseries\sffamily \thepieceindex \stepcounter{pieceindex}, %  \currentwq
            text=\fontsize{1.2ex}{1.2ex}\bfseries\sffamily \currentwq,
            markboard
        ]
    }
}
\newcommand{\pieceRolesTable}{
    \begin{tabular}{|l|}
        \hline
        \sympawn\ Pawn \\
        \hline
        \symknight\ Knight \\
        \hline
        \symbishop\ Bishop \\
        \hline
        \symrook\ Rook \\
        \hline
        \symqueen\ Queen \\
        \hline
        \symking\ King \\
        \hline
    \end{tabular}
}
\newcommand{\pieceColorsTable}{
    \begin{tabular}{|l|}
        \hline
        $\white$ White \\
        \hline
        $\black$ Black \\
        \hline
    \end{tabular}
}

\newcommand{\fs}[1]{\textsc{#1}}


\section{Feature sets (board encodings)}

To evaluate chess positions, we will use a neural network with an architecture explained in detail in the next chapter. In this chapter, we will build the one-dimensional input vector for such network, which can be described entirely by a feature set.

A feature set is a set built by a cartesian product of smaller sets of features, where each set extracts a different aspect of a position. Each tuple in the feature set corresponds to an element in the input vector, which will be set to $1$ if the aspects captured by the tuple is present in the position, and $0$ otherwise. If a tuple is present in a position, we say that the tuple is \textit{active}.

Let's consider some basic sets of features. The following sets encode positional information about the board:

\begin{center}
\begin{tabular}{cc}

$\begin{aligned}[t]
\fs{File} &= \{a, b, ..., h\} \\
\fs{Rank} &= \{1, 2, ..., 8\} \\
\fs{Square} &= \{a1, a2, ..., h8\}
\end{aligned}$

&

\raisebox{-10ex}{
\chessboard[
    tinyboard,
    showmover=false,
    pgfstyle={text},
    %text=\fontsize{1.2ex}{1.2ex}\bfseries\sffamily \thepieceindex \stepcounter{pieceindex}, %  \currentwq
    text=\fontsize{1.2ex}{1.2ex}\bfseries\sffamily \currentwq,
    markboard
]
}

\end{tabular}
\end{center}

And the following encode information about the pieces:

\begin{center}
$\begin{aligned}[t]
\fs{Role} &= \text{\{
    \sympawn\ Pawn,
    \symknight\ Knight,
    \symbishop\ Bishop,
    \symrook\ Rook,
    \symqueen\ Queen,
    \symking\ King\}}\textsuperscript{1} \\
\fs{Color} &= \text{\{\white\ White, \black\ Black\}}
\end{aligned}$
\end{center}

Since each set has to capture some information from the position, it must be stated explicitly. For example, consider the feature set $\fs{File}_{P} \times \fs{Color}_{P}$ where $P$ is \textit{any} piece in the board, meaning that the tuples $(file, color)$ that will be active are the ones where there is at least one piece in $file$ with the color $color$ (disregarding any other kind of information, like the piece's role). Another possible feature set could be $\fs{File}_{P} \times \fs{Role}_{P}$, with a similar interpretation. An illustration of the active features of these two feature sets for the same position is shown in Figure \ref{fig:active_features}.

\begin{figure}[h]
\centering
\label{fig:active_features}

\begin{tabular}{cc}
\raisebox{-7ex}{
\chessboard[
    tinyboard,
    showmover=false,
    hlabel=false,
    setwhite={kc3, nc2, pa2, Pd4},
    addblack={Kc8,bh7, pa7}
]
}

&

\begin{tabular}{|c|p{4cm}|p{4cm}|p{0cm}}
\cline{2-3}
\multicolumn{1}{c|}{} & \multicolumn{2}{c|}{\centering Feature set} \\
\cline{2-3}
\multicolumn{1}{c|}{} & \centering $\fs{File}_{P} \times \fs{Color}_{P}$ & \centering $\fs{File}_{P} \times \fs{Role}_{P}$ & \\
\cline{1-3}
Active features &
(a, \white), (a, \black), (c, \black), (c, \white), (d, \white), (h, \black) &
(a, \sympawn), (c, \symking), (c, \symknight), (d, \sympawn), (h, \symbishop) \\
\cline{1-3}
\end{tabular}

\end{tabular}

\caption{Active features of the feature sets $\fs{File}_{P} \times \fs{Color}_{P}$ and $\fs{File}_{P} \times \fs{Role}_{P}$ for the same position}
\end{figure}

\footnotetext[1]{The color of the pieces have no meaning in the definition. They are present for illustrative purposes.}

\subsection{Indexing}

The mapping between the tuples in a feature set and the elements in the input vector is computed using the order of the sets in the cartesian product and the size of each set, like strides in a multi-dimensional array. For this to work, each element in a set $S$ must correspond to a number between $0$ and $|S| - 1$. For example, the feature set $A \times B \times C$ has $|A| \times |B| \times |C|$ elements, and the tuple $(a, b, c)$ is mapped to the element $a \times |B| \times |C| + b \times |C| + c$.

\subsection{\mdseries\fs{Piece}}

This feature set is the most natural encoding for a chess position. There is a one-to-one mapping between pieces in the board and features:

\begin{center}
    $\fs{Piece} = \fs{Square}_{p} \times \fs{Role}_{p} \times \fs{Color}_{p}$ \\
    for every $p$ piece in the board
\end{center}


%\begin{table}[h]
%\centering
%\begin{tabular}{ccccc}
%\pieceBoard & $\bigtimes$ & \pieceRolesTable & %$\bigtimes$ & \pieceColorsTable \\
%\end{tabular}
%\caption{some caption}
%\end{table}


$\langle piece\_square, piece\_type, piece\_color \rangle$

$64*6*2=768$ features


\subsection{\mdseries\fs{Compact}}

This is one of the simplest feature sets.

$\langle piece\_rank, piece\_type, piece\_color \rangle \oplus \langle piece\_file, piece\_type, piece\_color \rangle$

$(8*6*2)*2=192$ features

\subsection{\mdseries\fs{Piece+Moves}}

\fs{HalfP} $\oplus \langle move\_from, move\_to \rangle$

$768 + 64*64=4864$ features

Not friendly to efficiently update the network. It is almost always better to do a full refresh on eval.

\subsection{\mdseries\fs{Half-King-Piece}}

$\fs{King-Piece} = \fs{Square}_{K} \times \fs{Piece}_{P}$ where $K$ is the king to move and $P$ is any \textit{non-king} piece

$\langle side\_king\_square, piece\_square, piece\_type, piece\_color \rangle$ excl. king

$64*64*5*2=40960$ features

There are variations to this feature set, such as \fs{HalfKAv2} or notably \fs{HalfKAv2\_hm} that is currently the latest feature set used by Stockfish 16.1. I will not consider them in this work.

known as "KP" in the literature

if we skip the king, you may be thinking where does it get the information about the other king's side, .... blabla arquitectura Half

\subsection{\mdseries\fs{Half-Relative(H$|$V$|$HV)King-Piece}}


$\langle side\_king\_file - piece\_file + 7, side\_king\_rank - piece\_rank + 7, piece\_type, piece\_color \rangle$ excl. king

$15*15*5*2=2250$ features (for HV)

only H or only V have $8*15*5*2=1200$ features


\subsection{\mdseries\fs{Half-Top(PP)}}

Statistical feature set, blabla, wasted features blabla


\subsection{Recap}

\begin{table}[h]
\centering
\begin{tabular}{|l|c|c|c|c|}
\hline
Feature set & Tuple & \# features \\
\hline
\fs{Piece} & $\fs{Square}_{P} \times \fs{Role}_{P} \times \fs{Color}_{P}$ & 768  \\
\fs{Compact} & asd & 192  \\
\fs{Piece+Moves} & asd & 4864 \\
\fs{King-Piece} & asd & 40,960 \\
\fs{RelativeHV-King-Piece} & asd & 2250  \\
\fs{TopPP} & asd & 64  \\
\hline
\end{tabular}
\caption{Comparison of feature sets}
\end{table}
