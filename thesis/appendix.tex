\appendix

\section{Appendix}

Runtime may be affected by other processes running on the machine, since it was my everyday computer. They are listed here for reference.

\subsection{Dataset}

asdasd

\subsection{Baseline runs}
\label{appendix:baseline}

\begin{table}[H]
\caption{Network architecture sweep results (baseline)\\(Games in tournament $\approx 40000$ each)}
\centering
\begin{adjustbox}{center}

\begin{tabular}{@{} cccccccc @{}} \toprule
\multirow{2}{*}{\bf Feature set} & \multicolumn{3}{c}{\bf Train hyperparams} & \multicolumn{2}{c@{}}{\bf Network} & \multirow{2}{*}{\makecell{\bf Val loss\\\textit{min}}} & \multirow{2}{*}{\makecell{\bf Runtime\\\textit{hh:mm:ss}}} \\
\cmidrule(lr){2-4} \cmidrule(l){5-6}
& \bf Batch & \bf LR & \bf Gamma & \bf L1 & \bf L2 & \\
\midrule
    \featureset{HV} & 16384 & 5e-04 & 0.99 & 256 & 32 & 0.00351 & 1:53:59 \\
\featureset{HV} & 16384 & 5e-04 & 0.99 & 256 & 64 & 0.00342 & 1:54:56 \\
\featureset{HV} & 16384 & 5e-04 & 0.99 & 256 & 128 & 0.00330 & 1:52:29 \\
\featureset{HV} & 16384 & 5e-04 & 0.99 & 256 & 256 & 0.00319 & 2:29:26 \\
\featureset{HV} & 16384 & 5e-04 & 0.99 & 512 & 32 & 0.00309 & 1:54:28 \\
\featureset{HV} & 16384 & 5e-04 & 0.99 & 512 & 64 & 0.00300 & 1:53:44 \\
\featureset{HV} & 16384 & 5e-04 & 0.99 & 512 & 128 & 0.00290 & 1:51:06 \\
\featureset{HV} & 16384 & 5e-04 & 0.99 & 512 & 256 & 0.00279 & 1:51:17 \\
\featureset{HV} & 16384 & 5e-04 & 0.99 & 1024 & 32 & 0.00268 & 2:15:18 \\
\featureset{HV} & 16384 & 5e-04 & 0.99 & 1024 & 64 & 0.00265 & 2:03:41 \\
\featureset{HV} & 16384 & 5e-04 & 0.99 & 1024 & 128 & 0.00257 & 2:06:39 \\
\featureset{HV} & 16384 & 5e-04 & 0.99 & 1024 & 256 & 0.00246 & 2:32:47 \\
\featureset{HV} & 16384 & 5e-04 & 0.99 & 2048 & 32 & 0.00241 & 3:11:56 \\
\featureset{HV} & 16384 & 5e-04 & 0.99 & 2048 & 64 & 0.00238 & 3:12:46 \\
\featureset{HV} & 16384 & 5e-04 & 0.99 & 2048 & 128 & 0.00234 & 3:29:07 \\
\featureset{HV} & 16384 & 5e-04 & 0.99 & 2048 & 256 & \textbf{0.00221} & 3:27:47 \\
\bottomrule \end{tabular}

\end{adjustbox}
\end{table}

\begin{figure}[H]
\centering
\makebox[\textwidth]{\includegraphics[width=\textwidth]{./dynamic/output/baseline_val_loss.pdf}}
\caption{Network architecture sweep validation loss over epochs (baseline)}
\end{figure}

%%%%%%%%%%%%%%%%%%%%%%%%%%%%%%%%%%%%%%%%%%%%%%%%%%%%%%%%%%%%%%%%%%%
%%%%%%%%%%%%%%%%%%%%%%%%%%%%%%%%%%%%%%%%%%%%%%%%%%%%%%%%%%%%%%%%%%%
%%%%%%%%%%%%%%%%%%%%%%%%%%%%%%%%%%%%%%%%%%%%%%%%%%%%%%%%%%%%%%%%%%%
%%%%%%%%%%%%%%%%%%%%%%%%%%%%%%%%%%%%%%%%%%%%%%%%%%%%%%%%%%%%%%%%%%%
%%%%%%%%%%%%%%%%%%%%%%%%%%%%%%%%%%%%%%%%%%%%%%%%%%%%%%%%%%%%%%%%%%%
%%%%%%%%%%%%%%%%%%%%%%%%%%%%%%%%%%%%%%%%%%%%%%%%%%%%%%%%%%%%%%%%%%%
%%%%%%%%%%%%%%%%%%%%%%%%%%%%%%%%%%%%%%%%%%%%%%%%%%%%%%%%%%%%%%%%%%%

\newpage
\subsection{Axes encoding}
\label{appendix:axes}

\subsubsection{Examples}
\label{appendix:axis_samples}

\begin{figure}[H]
\centering
\subfloat[\centering $\white$ White]{{\includegraphics[width=4.65cm]{../assets/results/piece_weights/white_bishop_weights.png} }}
\qquad
\subfloat[\centering $\white$ White]{{\includegraphics[width=4.65cm]{../assets/results/piece_weights/white_queen_weights.png} }}
\qquad
\subfloat[\centering $\white$ White]{{\includegraphics[width=4.65cm]{../assets/results/piece_weights/white_knight_weights.png} }} \\

\subfloat[\centering $\black$ Black]{{\includegraphics[width=4.65cm]{../assets/results/piece_weights/black_bishop_weights.png} }}
\qquad
\subfloat[\centering $\black$ Black]{{\includegraphics[width=4.65cm]{../assets/results/piece_weights/black_queen_weights.png} }}
\qquad
\subfloat[\centering $\black$ Black]{{\includegraphics[width=4.65cm]{../assets/results/piece_weights/black_knight_weights.png} }} \\

\caption{Weights of different neurons in the L1 layer, which are connected to features in \featureset{All} with different roles. The intensity represents the weight value, and the color represents the sign. The number is the feature index, specifically \featureset{VH} instead of \featureset{HV} (both are \featureset{All}), because it was prior to the first experiment. Refer to section \ref{sec:axis_encoding}.}
\end{figure}


\subsubsection{Preliminar runs}

    \begin{table}[H]
\caption{Axis feature sets preliminar runs}
\centering
\begin{adjustbox}{center}
\begin{tabular}{@{} cccc|cc @{}}
\toprule
\bf \multirow{2}{*}{Feature set} & \bf \multirow{2}{*}{Run} & \bf Val. loss & \bf Runtime & \bf Rating @ 192 & \bf Rating @ 256 \\
 &  & \textit{min} & \textit{hh:mm:ss} & \textit{TC=100ms/m} & \textit{TC=100ms/m} \\
\midrule
    \multirow{4}{*}{\featureset{D1} + \featureset{D2}} & 1 & \bf0.006707 & 1:44:25 & 2.1 $\pm$ 4.3 & \bf13.5 $\pm$ 4.6\\
 & 2 & 0.006716 & 1:45:46 & -3.9 $\pm$ 5.1 & -0.5 $\pm$ 5.0\\
 & 3 & 0.006729 & 1:47:58 & -4.7 $\pm$ 4.8 & -1.6 $\pm$ 5.3\\
 & 4 & 0.006721 & 1:51:24 & -0.9 $\pm$ 5.3 & -4.0 $\pm$ 5.0\\
\midrule
\multirow{4}{*}{\featureset{H} + \featureset{V}} & 1 & \bf0.005810 & 1:42:35 & -8.6 $\pm$ 5.2 & \bf9.5 $\pm$ 5.5\\
 & 2 & 0.005827 & 1:42:29 & -2.6 $\pm$ 5.4 & -6.5 $\pm$ 5.1\\
 & 3 & 0.005816 & 1:42:59 & 4.8 $\pm$ 4.8 & 2.4 $\pm$ 5.4\\
 & 4 & 0.005825 & 1:43:13 & -6.3 $\pm$ 4.9 & 7.4 $\pm$ 5.2\\
\midrule
\multirow{4}{*}{\featureset{H} + \featureset{V} + \featureset{D1} + \featureset{D2}} & 1 & \bf0.003885 & 2:26:05 & -14.3 $\pm$ 4.9 & -18.1 $\pm$ 4.3\\
 & 2 & 0.003907 & 2:27:30 & 7.2 $\pm$ 5.0 & \bf15.4 $\pm$ 4.7\\
 & 3 & 0.003905 & 2:27:35 & 0.1 $\pm$ 5.3 & 5.2 $\pm$ 4.4\\
 & 4 & 0.003906 & 2:45:19 & 5.7 $\pm$ 5.0 & -1.2 $\pm$ 4.5\\
\midrule
\multirow{4}{*}{\featureset{All}} & 1 & \bf0.003121 & 1:30:34 & -2.9 $\pm$ 4.7 & 4.6 $\pm$ 4.4\\
 & 2 & 0.003129 & 1:30:13 & -4.2 $\pm$ 5.0 & 10.1 $\pm$ 5.4\\
 & 3 & 0.003134 & 1:30:14 & -10.0 $\pm$ 5.2 & \bf10.4 $\pm$ 5.1\\
 & 4 & 0.003147 & 1:30:18 & -9.6 $\pm$ 5.0 & 1.6 $\pm$ 4.8\\
\midrule
\multirow{4}{*}{\featureset{All} + \featureset{D1} + \featureset{D2}} & 1 & 0.003093 & 2:06:54 & -5.0 $\pm$ 4.4 & 1.7 $\pm$ 4.5\\
 & 2 & \bf0.003087 & 2:12:30 & 8.6 $\pm$ 4.3 & \bf12.0 $\pm$ 4.7\\
 & 3 & \bf0.003087 & 2:26:29 & -3.1 $\pm$ 4.9 & 7.9 $\pm$ 3.9\\
 & 4 & 0.003095 & 2:38:25 & -6.1 $\pm$ 4.5 & -16.0 $\pm$ 4.4\\
\midrule
\multirow{4}{*}{\featureset{All} + \featureset{H} + \featureset{V}} & 1 & 0.003086 & 2:05:02 & 1.0 $\pm$ 4.8 & 9.0 $\pm$ 6.0\\
 & 2 & 0.003082 & 2:06:16 & \bf12.9 $\pm$ 4.8 & 7.1 $\pm$ 5.5\\
 & 3 & \bf0.003079 & 2:04:53 & -14.6 $\pm$ 5.1 & 2.3 $\pm$ 5.5\\
 & 4 & 0.003085 & 2:07:18 & -10.1 $\pm$ 4.9 & -7.6 $\pm$ 4.4\\
\midrule
\multirow{4}{*}{\featureset{All} + \featureset{H} + \featureset{V} + \featureset{D1} + \featureset{D2}} & 1 & 0.003071 & 2:49:23 & -18.7 $\pm$ 4.9 & 4.3 $\pm$ 4.6\\
 & 2 & 0.003052 & 2:42:18 & -6.6 $\pm$ 4.6 & -0.6 $\pm$ 4.8\\
 & 3 & 0.003067 & 2:44:26 & 6.5 $\pm$ 4.8 & \bf9.5 $\pm$ 4.6\\
 & 4 & \bf0.003050 & 2:44:34 & -2.9 $\pm$ 5.4 & 8.5 $\pm$ 4.6\\
\midrule
\multicolumn{6}{c}{\makecell{\textbf{Batch size}: 16384, \textbf{LR}: 5e-04, \textbf{Gamma}: 0.99, \textbf{L1}: 512, \textbf{L2}: 32
\\
A tournament was held for each feature set (8 networks), with 100ms per move.\\Opening book was UHO\_Lichess\_4852\_v1.epd. Each network played around 10000 games.\\Ratings computed using Ordo, relative to the average (rating=0 is the average).}} \\
\end{tabular}
\end{adjustbox}
\end{table}


\subsubsection{Final results}

    \begin{table}[H]
\caption{Axes feature sets final results}
\centering
\begin{adjustbox}{center}
\begin{tabular}{@{} cccc|cc @{}}
\toprule
\bf \multirow{2}{*}{Feature set} & \bf \multirow{2}{*}{Run} & \bf Val. loss & \bf Runtime & \bf Rating @ 192 & \bf Rating @ 256 \\
 &  & \textit{min} & \textit{hh:mm:ss} & \textit{TC=100ms/m} & \textit{TC=100ms/m} \\
\midrule
    \midrule
\multicolumn{6}{c}{\makecell{None}} \\
\end{tabular}
\end{adjustbox}
\end{table}




%%%%%%%%%%%%%%%%%%%%%%%%%%%%%%%%%%%%%%%%%%%%%%%%%%%%%%%%%%%%%%%%%%%
%%%%%%%%%%%%%%%%%%%%%%%%%%%%%%%%%%%%%%%%%%%%%%%%%%%%%%%%%%%%%%%%%%%
%%%%%%%%%%%%%%%%%%%%%%%%%%%%%%%%%%%%%%%%%%%%%%%%%%%%%%%%%%%%%%%%%%%
%%%%%%%%%%%%%%%%%%%%%%%%%%%%%%%%%%%%%%%%%%%%%%%%%%%%%%%%%%%%%%%%%%%
%%%%%%%%%%%%%%%%%%%%%%%%%%%%%%%%%%%%%%%%%%%%%%%%%%%%%%%%%%%%%%%%%%%
%%%%%%%%%%%%%%%%%%%%%%%%%%%%%%%%%%%%%%%%%%%%%%%%%%%%%%%%%%%%%%%%%%%
%%%%%%%%%%%%%%%%%%%%%%%%%%%%%%%%%%%%%%%%%%%%%%%%%%%%%%%%%%%%%%%%%%%

\newpage
\subsection{Pairwise axes runs}
\label{appendix:pairwise}

\begin{table}[H]
\caption{Pairwise feature sets sweep results\\(Games in tournament $\approx 24000$ each)}
\centering
\begin{adjustbox}{center}

\begin{tabular}{@{} ccccccccc @{}} \toprule
\multirow{2}{*}{\bf Feature set} &
\multicolumn{3}{c}{\bf Train hyperparams} &
\multicolumn{2}{c@{}}{\bf Network} &
\multirow{2}{*}{\makecell{\bf Val. loss\\\textit{min}}} &\multirow{2}{*}{\makecell{\bf Rating\\\textit{elo (rel. to \featureset{All})}}} &
\multirow{2}{*}{\makecell{\bf Runtime\\\textit{hh:mm:ss}}} \\
\cmidrule(lr){2-4} \cmidrule(l){5-6}
& \bf Batch & \bf LR & \bf Gamma & \bf L1 & \bf L2 & \\
\midrule
    \featureset{All} + \featureset{PH} & 16384 & 5e-04 & 0.99 & 512 & 32 & 0.00296 & \textbf{-0.5 $\pm$ 5.1} & 2:26:00 \\
\midrule
\featureset{All} + \featureset{PV} & 16384 & 5e-04 & 0.99 & 512 & 32 & 0.00303 & -14.1 $\pm$ 4.5 & 2:20:57 \\
\midrule
\featureset{All} + \featureset{PH} + \featureset{PV} & 16384 & 5e-04 & 0.99 & 512 & 32 & \textbf{0.00287} & -32.8 $\pm$ 4.7 & 3:32:35 \\
\bottomrule \end{tabular}

\end{adjustbox}
\end{table}


%%%%%%%%%%%%%%%%%%%%%%%%%%%%%

\newpage
\subsection{\texttt{emitPlainEntry} code}
\label{appendix:emitPlainEntry}

\lstset{
  %backgroundcolor=\color{gray!10},  
  basicstyle=\ttfamily,
  columns=fullflexible,
  breakatwhitespace=false,      
  breaklines=true,                
  captionpos=b,                    
  commentstyle=\color{green}, 
  extendedchars=true,              
  frame=single,                   
  keepspaces=true,             
  keywordstyle=\color{blue},      
  language=c++,                 
  numbers=none,                
  numbersep=5pt,                   
  numberstyle=\tiny\color{blue},
  rulecolor=\color{white},        
  showspaces=false,               
  showtabs=false,                 
  stepnumber=5,                  
  stringstyle=\color{red!60!blue},
  tabsize=3,                      
  title=\lstname                
}
\begin{lstlisting}
void emitPlainEntry(std::string& buffer, const TrainingDataEntry& plain)
{
    buffer += plain.pos.fen();
    buffer += ',';
    buffer += std::to_string(plain.score);
    buffer += ',';
    buffer += chess::uci::moveToUci(plain.pos, plain.move);
    buffer += '\n';
}
\end{lstlisting}

