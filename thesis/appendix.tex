\appendix

\section{Appendix}

Runtime may be affected by other processes running on the machine, since it was my personal computer. They are listed here for reference.

\subsection{Baseline runs}
\label{appendix:baseline}

\begin{tabular}{@{} cccccccc @{}} \toprule
\multirow{2}{*}{\bf Feature set} & \multicolumn{3}{c}{\bf Train hyperparams} & \multicolumn{2}{c@{}}{\bf Network} & \multirow{2}{*}{\makecell{\bf Val loss\\\textit{min}}} & \multirow{2}{*}{\makecell{\bf Runtime\\\textit{hh:mm:ss}}} \\
\cmidrule(lr){2-4} \cmidrule(l){5-6}
& \bf Batch & \bf LR & \bf Gamma & \bf L1 & \bf L2 & \\
\midrule
    \featureset{HV} & 16384 & 5e-04 & 0.99 & 256 & 32 & 0.00351 & 1:53:59 \\
\featureset{HV} & 16384 & 5e-04 & 0.99 & 256 & 64 & 0.00342 & 1:54:56 \\
\featureset{HV} & 16384 & 5e-04 & 0.99 & 256 & 128 & 0.00330 & 1:52:29 \\
\featureset{HV} & 16384 & 5e-04 & 0.99 & 256 & 256 & 0.00319 & 2:29:26 \\
\featureset{HV} & 16384 & 5e-04 & 0.99 & 512 & 32 & 0.00309 & 1:54:28 \\
\featureset{HV} & 16384 & 5e-04 & 0.99 & 512 & 64 & 0.00300 & 1:53:44 \\
\featureset{HV} & 16384 & 5e-04 & 0.99 & 512 & 128 & 0.00290 & 1:51:06 \\
\featureset{HV} & 16384 & 5e-04 & 0.99 & 512 & 256 & 0.00279 & 1:51:17 \\
\featureset{HV} & 16384 & 5e-04 & 0.99 & 1024 & 32 & 0.00268 & 2:15:18 \\
\featureset{HV} & 16384 & 5e-04 & 0.99 & 1024 & 64 & 0.00265 & 2:03:41 \\
\featureset{HV} & 16384 & 5e-04 & 0.99 & 1024 & 128 & 0.00257 & 2:06:39 \\
\featureset{HV} & 16384 & 5e-04 & 0.99 & 1024 & 256 & 0.00246 & 2:32:47 \\
\featureset{HV} & 16384 & 5e-04 & 0.99 & 2048 & 32 & 0.00241 & 3:11:56 \\
\featureset{HV} & 16384 & 5e-04 & 0.99 & 2048 & 64 & 0.00238 & 3:12:46 \\
\featureset{HV} & 16384 & 5e-04 & 0.99 & 2048 & 128 & 0.00234 & 3:29:07 \\
\featureset{HV} & 16384 & 5e-04 & 0.99 & 2048 & 256 & \textbf{0.00221} & 3:27:47 \\
\bottomrule \end{tabular}


\subsection{Axis encoding examples}
\label{appendix:axis_samples}

\begin{figure}[h]
\centering
\subfloat[\centering $\white$ White]{{\includegraphics[width=4.65cm]{../assets/results/piece_weights/white_bishop_weights.png} }}
\qquad
\subfloat[\centering $\white$ White]{{\includegraphics[width=4.65cm]{../assets/results/piece_weights/white_queen_weights.png} }}
\qquad
\subfloat[\centering $\white$ White]{{\includegraphics[width=4.65cm]{../assets/results/piece_weights/white_knight_weights.png} }} \\

\subfloat[\centering $\black$ Black]{{\includegraphics[width=4.65cm]{../assets/results/piece_weights/black_bishop_weights.png} }}
\qquad
\subfloat[\centering $\black$ Black]{{\includegraphics[width=4.65cm]{../assets/results/piece_weights/black_queen_weights.png} }}
\qquad
\subfloat[\centering $\black$ Black]{{\includegraphics[width=4.65cm]{../assets/results/piece_weights/black_knight_weights.png} }} \\

\caption{Weights of different neurons in the L1 layer, which are connected to features in \featureset{Piece} with different roles. The intensity represents the weight value, and the color represents the sign. The number is the feature index, specifically \featureset{VH} instead of \featureset{HV} (both are \featureset{Piece}), because it was prior to the first experiment. Refer to section \ref{sec:axis_encoding}.}
\end{figure}
