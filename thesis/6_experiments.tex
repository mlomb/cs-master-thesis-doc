\section{Experiments and results}


Most of the experiments will use the eval training method.

Dimensions:

\begin{itemize}
\item Feature set
\item L1-L4 sizes
\item Training method
\end{itemize}

Metrics:

\begin{itemize}
\item Puzzle accuracy
\item Relative ELO performance
\item Inference performance (infs/s)
\end{itemize}

\subsection{Baselines}

P and KP
Network size?
Inference time?

\subsection{Axis relevance}

H, V, D1, D2
Compact

Mirar si sacando informacion en un eje mejora/empeora. La idea es ver si la informacion vertical es mas importante que la horizontal. Ver si incluir diagonales ayuda.

\subsection{Symmetry}

Medir el impacto de agregar simetría al fs. Red mas chica, inf mas rapida, mejor perf?

\fs{Half-Relative(H|V|HV)King-Piece}?

\subsection{Piece movement}

Intentar capturar los patrones que se ven en P, asi se pueden reconocer patrones mas complejos.

\fs{Piece-Move}

Bad perf.

\subsection{Statistical features}

Top P

Hacer un subset de \fs{PP} (589824).

\begin{itemize}
\item Destilar?
\item Probar si es lo mismo quedarse con el TOP K de las mas comunes o con las que dice el performance.
\item Catboost? PCA?
\end{itemize}

\subsection{Human behavior}

PQR human behaviour. Medir estilo Maia. comparar? no va a ser tan bueno.
