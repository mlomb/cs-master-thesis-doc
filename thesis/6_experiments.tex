\newcommand{\depiction}[1]{\parbox{0.7cm}{\includegraphics[height=0.7cm]{../assets/depictions/#1.pdf}}}
\newcommand{\depictionSM}[1]{\parbox{0.6cm}{\includegraphics[height=0.6cm]{../assets/depictions/#1.pdf}}}


\section{Experiments and results}

Now that the engine, the tools and the methodology are defined, we can proceed to the experiments. Experiments will be divided in three sections: motivation, experiment and results. The motivation will explain why I think the experiment is relevant and present possible hypothesis. The experiment will describe configurations to train different models, how they will be evaluated and what are my expectations. The results will present the data, explain whether my hypothesis was correct or not and give a brief conclusion. \\

Every model's training configuration is defined by the following variables:

\begin{itemize}
\item \textbf{Feature set}: Determinates the encoding of the position, and thus the number of inputs of the model. It conditions which patterns the network can learn. Experimenting with this is the main focus of this thesis.

\item \textbf{Network architecture}: The size of each layer in the network. The first layer (L1) is the feature transformer and it is efficiently updated. The following layer (L2) should be tiny due the NNUE architecture. The size of the model (its complexity) roughly determinates how many patterns the network can learn.

\item \textbf{Dataset}: The positions to train on. The dataset used is explained in detail in chapter 5. In summary, there are 48.5 billion positions to train on and the dataset remains constant across all runs. About 5 million positions are used for validation.

% no me gusta la palabra computed...
\item \textbf{Training method}: Can choose to use either score targets or PQR triplets. This determinates the format of the samples as well as the loss function. All experiments will train using score targets, unless specified. Methods were explained in detail in chapter 5.

\item \textbf{Training hyperparameters}: The usual machine learning hyperparameters for training, such as batch size, learning rate and scheduler. I used the same epoch size used in Stockfish, where each epoch is 100 million positions. Each training run will last for 256 epochs, which means the network is trained in 25.6 billion positions (recall that some of the original 48.5 billion dataset are skipped).
\end{itemize}

Once training is completed, the models will be evaluated depending on the experiment. To assess the performance of a model or to compare a set of models, the following indicators are used:

\begin{itemize}
\item \textbf{Loss}: The training and validation loss are used to detect overfitting and other possible problems. It can't be used to measure the performance of a model. Bigger models must have much better predictions to outweight the cost of having slower inferences and thus less node visits. It's a tradeoff.

\item \textbf{Puzzle accuracy}: The percentage of moves correctly predicted by the engine in Lichess puzzles. Each puzzle may contain multiple moves, and the engine has 100ms per move. Since the engine is not that strong, it does not solve 100\% of puzzles like many other engines do, so I expect differences in this metric to be good indicators. A small set of puzzles is used during training as (a very bad) proxy for the engine's strength, to have early insight of the strength and to detect catastrophic failures that did arise. A bigger set of 85000 puzzles is used after training.

\item \textbf{Relative ELO rating}: A tournament is played between different models to determine their relative strength. Ordo is used to compute the ELO of each model based on the results of the tournament. This is the most important metric, as it is the most reliable way to compare the strength of engines.

% \item \textbf{Inference performance (infs/s)}:

\item \textbf{Training duration}: The amount of time it takes to train a model. This is a one time operation and it does not affect the performance of a model. However, it does condition which and how many experiments I can run.
\end{itemize}

All networks that are not in the first experiment (the baseline), are trained 4 times and a tournament is played between the epoch 192 and 256 of each network (8 networks in total). I have observed a difference of 30 elo points between runs, so this step is crucial to have sensible results. In the appendix are the results of each run and tournament.


\begin{frame}
\frametitle{Setup de training}
Recapitulando... ¿Qué hay que definir para entrenar una red? \pause
\begin{itemize}
\item (variable) \textbf{Feature set}: determina el encoding \pause
\item \checkmark\ \textbf{Dataset}: datos de entrenamiento \pause
\item \checkmark\ \textbf{Método de entrenamiento}: PQR/target scores; determina el formato de las muestras y la loss function  \pause
\item ? \textbf{Arquitectura de la red}: el tamaño de cada capa; $L_1$ y $L_2$ \pause
\item ? \textbf{Hiperparámetros}: learning rate, batch size, epochs, etc.
\end{itemize}
\end{frame}

\begin{frame}
\frametitle{Setup de evaluación}
¿Cómo evalúo el performance de una red entrenada? \pause
\begin{itemize}
\item \textbf{Loss} (train y val.): indica la calidad de las predicciones.
\begin{itemize}
    \item Permite detectar overfitting y otros problemas \pause
\end{itemize}
\item \textbf{Puzzle accuracy}: porcentaje de movimientos acertados en puzzles de Lichess.
\begin{itemize}
    \item Sólo hay un movimiento correcto
    \item Proxy (muy malo) de la fuerza de la red \pause
\end{itemize}
\item \textbf{Elo relativo}: la medida más común para comparar engines.
\begin{itemize}
    \item Se realizan torneos de 100ms por movimiento
    \item El elo es calculado a partir de Ordo
\end{itemize}
\end{itemize}
\end{frame}

\begin{frame}
\frametitle{Baseline: motivación}
Busco fijar el setup de entrenamiento con valores razonables \pause
Queda por determinar\dots
\begin{itemize}
\item La arquitectura de la red: $L_1$ y $L_2$
\item Los hiperparámetros
\end{itemize}
\end{frame}

\begin{frame}
\frametitle{Baseline: hiperparámetros}
Los hiperparámetros fueron seleccionados en base al trainer oficial de Stockfish:
\begin{itemize}
\item \textbf{Learning rate}: 0.0005
\item \textbf{Exponential decay}: 0.99
\item \textbf{Batch size}: 16384
\item \textbf{Epoch size}: 100 million
\begin{itemize}
    \item cada epoch realiza 6104 batches
\end{itemize}
\item \textbf{Epochs}: 256
\begin{itemize}
    \item cada run observa \textit{25.6 billion} samples
\end{itemize}
\end{itemize}
\end{frame}


\begin{frame}
\frametitle{Baseline: experimento}
Sólo queda buscar parámetros $L1$ y $L2$ razonables. Realizo una búsqueda en grilla con:
\begin{itemize}
\item $L1$ $\in \{256, 512, 1024, 2048\}$
\item $L2$ $\in \{32, 64, 128, 256\}$
\end{itemize}
El feature set a utilizar es \featureset{All}[768].
\end{frame}

\begin{frame}
\frametitle{Baseline: resultados}
\begin{figure}
\centering
\makebox[\textwidth]{\includegraphics[width=0.93\textwidth]{../thesis/dynamic/output/baseline_heatmaps.pdf}}
\end{figure}
\end{frame}

\begin{frame}
\frametitle{Baseline: conclusión}
\begin{itemize}
\item \textbf{L2=32}. El performance cae dramáticamente si L2 aumenta, utilizo el más bajo.
\begin{itemize}
    \item Sería buena idea probar valores más chicos de L2. \pause
\end{itemize}
\item \textbf{L1=512}. Es el mejor valor para L2=64 y L2=128, y en margen de error para L2=32.
\begin{itemize}
    \item Además es el más rápido de entrenar.
\end{itemize}
\end{itemize}
\end{frame}


\begin{frame}
\frametitle{Axis encoding: motivación}

\begin{figure}[h]
\centering
\subfloat[\centering $\white$ White]{{\includegraphics[width=4.4cm]{../assets/results/piece_weights/white_rook_weights.png} }}%
\qquad
\subfloat[\centering $\black$ Black]{{\includegraphics[width=4.4cm]{../assets/results/piece_weights/black_rook_weights.png} }}%
\caption{Weights of \textbf{a neuron} in the L1 layer, which are connected to features in \featureset{All} where the role is $\rook$ Rook. The intensity represents the weight value, and the color represents the sign (although not relevant).}
\label{fig:rook_weights}
\end{figure}

\end{frame}


\begin{frame}
\frametitle{Pairwise axes: motivación}

\storechessboardstyle{smallvert}{
    tinyboard,
    maxfield=a8,
    showmover=false,
    hmargin=false,
    hlabel=false,
    boardfontsize=15pt,
}

\newcommand{\raiseby}{-11.5ex}

\begin{figure}
\centering

\begin{tabular}{ccc}

\raisebox{\raiseby}{\chessboard[
    style=smallvert,
    clearboard,
    addblack={Ra7},
    addwhite={na3,pa4},
]}
\raisebox{\raiseby}{\chessboard[
    style=smallvert,
    clearboard,
    addblack={Ra8},
    addwhite={na2,pa3},
]}
\raisebox{\raiseby}{\chessboard[
    style=smallvert,
    clearboard,
    addblack={Ra6},
    addwhite={na3,pa4},
]}
\raisebox{\raiseby}{\chessboard[
    style=smallvert,
    clearboard,
    addblack={Ra5},
    addwhite={na1,pa3},
]}
\raisebox{\raiseby}{\chessboard[
    style=smallvert,
    clearboard,
    addblack={Ra6},
    addwhite={na2,pa3},
]}

$\hdots$

&
$ \rightarrow$
&

\raisebox{-9.5ex}{\chessboard[
    blackfieldcolor=white,
    blackfieldmaskcolor=white,
    maxfield=a8,
    style=smallvert,
    vlabel=false,
    border=false,
    trim=false,
    opacity=0.6,
    clearboard,
    addblack={Ra6},
    addwhite={na2,pa4},
    %
    color=red,
    shortenend=1.88ex,shortenstart=1.88ex, % espacio
    padding=-1ex,
    markstyle=leftborder,
    linewidth=0.4ex,
    markregion=a4-a6,
    linewidth=1.6ex,
    pgfstyle=circle,
    markfields={a4,a6},
    %
    color=blue,
    shortenend=1.88ex,shortenstart=1.88ex, % espacio
    padding=-1ex,
    markstyle=leftborder,
    linewidth=0.4ex,
    markregion=a2-a4,
    linewidth=1.6ex,
    pgfstyle=circle,
    markfields={a2,a4},
    %
]}

\\

\makecell{Configuraciones distintas,\\situaciones similares} &  & \makecell{Las mismas dos features\\(par rojo y par azul)}

\end{tabular}
\end{figure}
\end{frame}

\begin{frame}
\frametitle{Pairwise axes: motivación}
Comparando con el experimento anterior, es más específico en vez de más general:
\begin{center}
\enquote{\textit{there is a $\white$ White $\rook$ Rook in the 4th rank}} \\
vs. \\
\enquote{\textit{there is a $\black$ Black $\rook$ Rook next to a $\white$ White $\sympawn$ Pawn in the \enquote{a} file}}
\end{center}
\end{frame}

\begin{frame}
\frametitle{Pairwise axes: experimento}
\begin{table}
\small
\centering
\begin{adjustbox}{max width=\textwidth}
\begin{tabular}{ccccc}
\toprule
\bf D. & \bf \makecell{Block\\name} & \bf Definition & \bf \makecell{Num. of\\features} \\
\toprule
\depiction{PH} & PH & \makecell{
\vspace{0.2cm}
$(\featureset{Ranks} \times (\featureset{Roles} \times \featureset{Colors}) \times (\featureset{Roles} \times \featureset{Colors}))_{P}$ \\
P($\langle r, r_1, c_1, r_2, c_2 \rangle$): there is a piece in rank $r$ with role $r_1$\\ and color $c_1$ to the left of a piece with role $r_2$ and color $c_2$
} & 1152 \\
\toprule
\depiction{PV} & PV & \makecell{
\vspace{0.2cm}
$(\featureset{Files} \times (\featureset{Roles} \times \featureset{Colors}) \times (\featureset{Roles} \times \featureset{Colors}))_Q$ \\
Q($\langle f, r_1, c_1, r_2, c_2 \rangle$): there is a piece in file $f$ with role $r_1$\\ and color $c_1$ below a piece with role $r_2$ and color $c_2$
} & 1152 \\
\bottomrule
\end{tabular}
\end{adjustbox}
\end{table}
\end{frame}

\begin{frame}
\frametitle{Pairwise axes: experimento}
\begin{figure}[h]
\centering
\begin{adjustbox}{max width=\textwidth}
\begin{tabular}{cc}

\raisebox{-7ex}{\chessboard[
    setfen=2r4k/p5p1/Kpqp3p/8/1PP2Q2/P2P1RP1/8/8 b - - 12 45,
    showmover=false,
    opacity=0.6,
    %
    % TEMPLATE HORIZONTAL
    %color=red,
    %shortenend=1.88ex,shortenstart=1.88ex, % espacio
    %padding=-1ex,
    %markstyle=topborder,
    %linewidth=0.4ex,
    %markregion=d6-d3,
    %linewidth=1.6ex,
    %pgfstyle=circle,
    %markfields={d6,d3},
    %
    color=red,
    shortenend=1.88ex,shortenstart=1.88ex, % espacio
    padding=-1ex,
    markstyle=topborder,
    linewidth=0.4ex,
    markregion=a3-d3,
    linewidth=1.6ex,
    pgfstyle=circle,
    markfields={a3,d3},
    %
    color=red,
    shortenend=1.88ex,shortenstart=1.88ex, % espacio
    padding=-1ex,
    markstyle=topborder,
    linewidth=0.4ex,
    markregion=f3-g3,
    linewidth=1.6ex,
    pgfstyle=circle,
    markfields={f3,g3},
    %
    color=blue,
    shortenend=1.88ex,shortenstart=1.88ex, % espacio
    padding=-1ex,
    markstyle=topborder,
    linewidth=0.4ex,
    markregion=d3-f3,
    linewidth=1.6ex,
    pgfstyle=circle,
    markfields={d3,f3},
    %
    color=red,
    shortenend=1.88ex,shortenstart=1.88ex, % espacio
    padding=-1ex,
    markstyle=topborder,
    linewidth=0.4ex,
    markregion=b4-c4,
    linewidth=1.6ex,
    pgfstyle=circle,
    markfields={b4,c4},
    %
    color=blue,
    shortenend=1.88ex,shortenstart=1.88ex, % espacio
    padding=-1ex,
    markstyle=topborder,
    linewidth=0.4ex,
    markregion=c4-f4,
    linewidth=1.6ex,
    pgfstyle=circle,
    markfields={c4,f4},
    %
    color=red,
    shortenend=1.88ex,shortenstart=1.88ex, % espacio
    padding=-1ex,
    markstyle=topborder,
    linewidth=0.4ex,
    markregion=a6-b6,
    linewidth=1.6ex,
    pgfstyle=circle,
    markfields={a6,b6},
    %
    color=red,
    shortenend=1.88ex,shortenstart=1.88ex, % espacio
    padding=-1ex,
    markstyle=topborder,
    linewidth=0.4ex,
    markregion=c6-d6,
    linewidth=1.6ex,
    pgfstyle=circle,
    markfields={c6,d6},
    %
    color=blue,
    shortenend=1.88ex,shortenstart=1.88ex, % espacio
    padding=-1ex,
    markstyle=topborder,
    linewidth=0.4ex,
    markregion=b6-c6,
    linewidth=1.6ex,
    pgfstyle=circle,
    markfields={b6,c6},
    %
    color=blue,
    shortenend=1.88ex,shortenstart=1.88ex, % espacio
    padding=-1ex,
    markstyle=topborder,
    linewidth=0.4ex,
    markregion=d6-h6,
    linewidth=1.6ex,
    pgfstyle=circle,
    markfields={d6,h6},
    %
    color=red,
    shortenend=1.88ex,shortenstart=1.88ex, % espacio
    padding=-1ex,
    markstyle=topborder,
    linewidth=0.4ex,
    markregion=a7-g7,
    linewidth=1.6ex,
    pgfstyle=circle,
    markfields={a7,g7},
    %
    color=blue,
    shortenend=1.88ex,shortenstart=1.88ex, % espacio
    padding=-1ex,
    markstyle=topborder,
    linewidth=0.4ex,
    markregion=c8-h8,
    linewidth=1.6ex,
    pgfstyle=circle,
    markfields={c8,h8},
]}

&

\raisebox{-7ex}{\chessboard[
    setfen=2r4k/p5p1/Kpqp3p/8/1PP2Q2/P2P1RP1/8/8 b - - 12 45,
    showmover=false,
    opacity=0.6,
    %
    % TEMPLATE VERTICAL
    %color=red,
    %shortenend=1.88ex,shortenstart=1.88ex, % espacio
    %padding=-1ex,
    %markstyle=leftborder,
    %linewidth=0.4ex,
    %markregion=d6-d3,
    %linewidth=1.6ex,
    %pgfstyle=circle,
    %markfields={d6,d3},
    %
    color=red,
    shortenend=1.88ex,shortenstart=1.88ex, % espacio
    padding=-1ex,
    markstyle=leftborder,
    linewidth=0.4ex,
    markregion=a3-a6,
    linewidth=1.6ex,
    pgfstyle=circle,
    markfields={a3,a6},
    %
    color=blue,
    shortenend=1.88ex,shortenstart=1.88ex, % espacio
    padding=-1ex,
    markstyle=leftborder,
    linewidth=0.4ex,
    markregion=a6-a7,
    linewidth=1.6ex,
    pgfstyle=circle,
    markfields={a6,a7},
    %
    color=red,
    shortenend=1.88ex,shortenstart=1.88ex, % espacio
    padding=-1ex,
    markstyle=leftborder,
    linewidth=0.4ex,
    markregion=b4-b6,
    linewidth=1.6ex,
    pgfstyle=circle,
    markfields={b4,b6},
    %
    color=red,
    shortenend=1.88ex,shortenstart=1.88ex, % espacio
    padding=-1ex,
    markstyle=leftborder,
    linewidth=0.4ex,
    markregion=c4-c6,
    linewidth=1.6ex,
    pgfstyle=circle,
    markfields={c4,c6},
    %
    color=blue,
    shortenend=1.88ex,shortenstart=1.88ex, % espacio
    padding=-1ex,
    markstyle=leftborder,
    linewidth=0.4ex,
    markregion=c6-c8,
    linewidth=1.6ex,
    pgfstyle=circle,
    markfields={c6,c8},
    %
    color=red,
    shortenend=1.88ex,shortenstart=1.88ex, % espacio
    padding=-1ex,
    markstyle=leftborder,
    linewidth=0.4ex,
    markregion=d3-d6,
    linewidth=1.6ex,
    pgfstyle=circle,
    markfields={d3,d6},
    %
    color=red,
    shortenend=1.88ex,shortenstart=1.88ex, % espacio
    padding=-1ex,
    markstyle=leftborder,
    linewidth=0.4ex,
    markregion=f3-f4,
    linewidth=1.6ex,
    pgfstyle=circle,
    markfields={f3,f4},
    %
    color=red,
    shortenend=1.88ex,shortenstart=1.88ex, % espacio
    padding=-1ex,
    markstyle=leftborder,
    linewidth=0.4ex,
    markregion=g3-g7,
    linewidth=1.6ex,
    pgfstyle=circle,
    markfields={g3,g7},
    %
    color=red,
    shortenend=1.88ex,shortenstart=1.88ex, % espacio
    padding=-1ex,
    markstyle=leftborder,
    linewidth=0.4ex,
    markregion=h6-h8,
    linewidth=1.6ex,
    pgfstyle=circle,
    markfields={h6,h8},
]}


\\

\makecell{\depiction{PH} Pairwise horizontal (\featureset{PH})} &
\makecell{\depiction{PV} Pairwise vertical (\featureset{PV})}

\end{tabular}
\end{adjustbox}
\end{figure}
\end{frame}


\begin{frame}
\frametitle{Pairwise axes: experimento}
Los feature sets a entrenar son:
\begin{itemize}
\item \featureset{All} $\oplus$ \featureset{PH} (1920 features)
\item \featureset{All} $\oplus$ \featureset{PV} (1920 features)
\item \featureset{All} $\oplus$ \featureset{PH} $\oplus$ \featureset{PV} (3072 features)
\end{itemize}
\end{frame}

\begin{frame}
\frametitle{Pairwise axes: resultados}
\begin{table}
\centering
\begin{tabular}{ccccc}
\toprule
\bf Feature set  & \bf \makecell{Number\\of features} & \makecell{\bf Val. loss\\\textit{min}} & \makecell{\bf Rating\\\textit{elo (rel. to \featureset{All})}} \\
\toprule
\featureset{All} (reference) & 768 & 0.003134 & \textbf{0.0} \\
\midrule
\featureset{All} $\oplus$ \depiction{PH} & 1920 & 0.003033 & -38.2 $\pm$ 4.8 \\
\midrule
\featureset{All} $\oplus$ \depiction{PV} & 1920 & 0.002946 & -8.4 $\pm$ 5.0 \\
\midrule
\featureset{All} $\oplus$ \depiction{PH} $\oplus$ \depiction{PV} & 3072 & \textbf{0.002868} & -37.6 $\pm$ 4.9 \\
\bottomrule
\end{tabular}
\end{table}
\begin{itemize}
    \item Reducir el número de pairs puede llevar a una mejora por sobre \featureset{All} (ej. $\sympawn$)
\end{itemize}
\end{frame}


\noindent\rule{\textwidth}{1pt}

\vspace{0.2cm}
Up to this point, I have been trying to encode the position of the pieces in different or smarter ways, with no avail. It may seem that the network is able to extract all the information it needs from the most basic \featureset{All} feature set. Making the information available in another form makes no difference, as opposed to what I originally thought.

Further experiments will focus on features not related to the position of the pieces, but to other aspects of the game, inspired by hand crafted evaluations.

\noindent\rule{\textwidth}{1pt}


\subsection{Mobility}

\textbf{Motivation.} Mobility in chess is a measure of the available moves a player can make in a given position. The idea is that if a player has more available moves, the position is stronger. In \cite{slater:1950} it was shown that there is a strong correlation between a player's mobility and the number of games won. This metric has been used extensively in hand crafted evaluations, and I propose to include this information as features for the neural network. \\

\textbf{Experiment.} There are two ways to go about encoding mobility:

\begin{itemize}
\item \textbf{Bitsets (per piece type):} Provide the exact squares each piece type can move to. The number of features would be $64 * 6 * 2 = 768$. The problem with this approach is not the amount of features, but the number of updates to the accumulator per move is very high, which slows down the search.

\begin{figure}[H]
\centering

\begin{tabular}{ccccc}

\raisebox{-7ex}{\chessboard[
    setfen=r5k1/1b1p1ppp/p7/1p1Q4/2p1r3/PP4Pq/BBP2b1P/R4R1K w - - 0 20,
    tinyboard,
    showmover=false,
]}
&

\raisebox{-7ex}{\chessboard[
    tinyboard,
    showmover=false,
    setwhite={ba2,bb2},
    pgfstyle=color,
    opacity=0.8,
    color=blue,
    markfield={b1,c1,c3,d4,e5,f6,g7}
]}

&

\raisebox{-7ex}{\chessboard[
    tinyboard,
    showmover=false,
    addblack={Bb7,Bf2},
    pgfstyle=color,
    opacity=0.8,
    color=blue,
    markfield={c8,c6,d5,a7,b6,c5,d4,e3,e1,g1,g3}
]}

&

\raisebox{-7ex}{\chessboard[
    tinyboard,
    showmover=false,
    setwhite={qd5},
    pgfstyle=color,
    opacity=0.8,
    color=blue,
    markfield={d6,d7,e6,f7,e5,f5,g5,h5,e4,d4,d3,d2,d1,c4,c5,b5,c6,b7}
]}

& $\hdots$

\\

Board &
\makecell{\white White\\\symbishop\ Bishop} &
\makecell{\black Black\\\symbishop\ Bishop} &
\makecell{\white White\\\symqueen\ Queen}

\end{tabular}
\end{figure}


\item \textbf{Counts (per piece type):} Use the number of available moves per piece type as features. This means having a feature for each possible count value, which are a lot less feature updates. To find which values to include as features I computed the total mobility for each piece role in 2 billion boards, shown in figure \ref{fig:mobility}. From the data we can extract the range of values to use as features:

\begin{table}[H]
\centering
\begin{tabular}{c|c|c}
\toprule
\textbf{Piece role} & \textbf{Min} & \textbf{Max} \\
\midrule
\sympawn\ Pawn & 0 & 8+ \\
\symknight\ Knight & 0 & 15+ \\
\symbishop\ Bishop & 0 & 16+ \\
\symrook\ Rook & 0 & 25+ \\
\symqueen\ Queen & 0 & 25+ \\
\symking\ King & 0 & 8 \\
\bottomrule
\end{tabular}
\end{table}

\begin{figure}[H]
\centering
\makebox[\textwidth]{\includegraphics[width=\textwidth]{./dynamic/output/mobility.pdf}}
\caption{Total mobility values for each piece on the board. Computed using 2 billion boards. The value 0 for the \symknight\ Knight, \symbishop\ Bishop, \symrook\ Rook and \symqueen\ Queen has been excluded from the plot, as it is very common.}
\label{fig:mobility}
\end{figure}
\end{itemize}

Each approach was implemented as a block:

\begin{table}[H]
\caption{Mobility feature blocks}
\label{tab:mobility_blocks}
\centering

\begin{tabular}{ccc}
\toprule
\bf Block name & \bf Definition & \bf Number of features \\
\toprule
MB & \makecell{
\vspace{0.2cm}
($\featureset{Squares} \times \featureset{Roles} \times \featureset{Colors})_P$ \\
P($\langle s, r, c \rangle$): there is a piece of role $r$\\ and color $c$ that \textbf{can move to} square $s$
} & 768 \\
\toprule
MC & \makecell{
\vspace{0.2cm}
$(\{0, 1, \hdots\} \times \featureset{Roles} \times \featureset{Colors})_{P}$\\
P($\langle m, r, c \rangle$): the value of mobility for\\
a piece of role $r$ and color $c$ is $m$
} & 206 \\
\bottomrule
\end{tabular}
\end{table}

The blocks will be combined with the \featureset{All} feature set. Neither of the blocks can be used alone, since they do not carry the information to deduce every piece on the board (trivially).

The feature sets to be trained and evaluated are \featureset{All} $\oplus$ \featureset{MB} (1536 features) and \featureset{All} $\oplus$ \featureset{MC} (974 features). Like prior experiments, a network will be trained for each feature set and evaluated in a tournament. \\

\textbf{Results.} The results in table \ref{tab:mobility_results} show that the blocks did not provide features of enough quality to improve the network predictions enough to overcome the cost of making more feature updates. I have underestimated the cost of a feature update, that was not a problem in previous experiments. The block \featureset{MB} has, in average, 10.03 feature updates per move and the block \featureset{MC} has 3.82 (see Appendix \ref{appendix:fs}). In comparison, the \featureset{All} feature set has 1.58 feature updates per move. Even though the block \featureset{MB} has almost 3 times the number of feature updates, it has a better performance than \featureset{MC}. This is attributed to having a 7\% lower loss, which compensates the cost of the updates. \\

\begin{table}[H]
\caption{Mobility encodings results}
\label{tab:mobility_results}
\centering

\begin{tabular}{ccccc}
\toprule
\bf Feature set  & \bf \makecell{Number\\of features} & \makecell{\bf Val. loss\\\textit{min}} & \makecell{\bf Rating\\\textit{elo (rel. to \featureset{All})}} \\
\toprule
\featureset{All} (reference) & 768 & 0.003134 & \textbf{0.0} \\
\midrule
\featureset{All} $\oplus$ \featureset{MB} & 1536 & 0.002824 & -260.9 $\pm$ 5.4 \\
\midrule
\featureset{All} $\oplus$ \featureset{MC} & 974 & 0.003032 & -280.9 $\pm$ 5.6 \\
\bottomrule
\end{tabular}
\end{table}


\begin{frame}
\frametitle{PQR}
:)
\end{frame}


%\subsection{Active neurons}
%medir si hay feature sets que no usen neuronas, que esto disparo el uso de HalfTopK
%average number of features enabled by feature set (cantidad y porcentaje)
%measure updates per move average and refreshes average per FS
%[ESTO PONERLO EN EL APPENDIX]


%%%%%%%%%%%%%%%%%%%%%%%%%%%%%%%%%%%%%%%%%%%
%%%%%%%%%%%%%%%%%%%%%%%%%%%%%%%%%%%%%%%%%%%
%%%%%%%%%%%%%%%%%%%%%%%%%%%%%%%%%%%%%%%%%%%
%%%%%%%%%%%%%%%%%%%%%%%%%%%%%%%%%%%%%%%%%%%
%%%%%%%%%%%%%%%%%%%%%%%%%%%%%%%%%%%%%%%%%%%
%%%%%%%%%%%%%%%%%%%%%%%%%%%%%%%%%%%%%%%%%%%
%%%%%%%%%%%%%%%%%%%%%%%%%%%%%%%%%%%%%%%%%%%
%%%%%%%%%%%%%%%%%%%%%%%%%%%%%%%%%%%%%%%%%%%
%%%%%%%%%%%%%%%%%%%%%%%%%%%%%%%%%%%%%%%%%%%
%%%%%%%%%%%%%%%%%%%%%%%%%%%%%%%%%%%%%%%%%%%
%%%%%%%%%%%%%%%%%%%%%%%%%%%%%%%%%%%%%%%%%%%


% \noindent\makebox[\linewidth]{\rule{\paperwidth}{0.4pt}}
% 
% Esto de abajo son notas. No se si hacerlo o no... quizas es mucho
% 
% \noindent\makebox[\linewidth]{\rule{\paperwidth}{0.4pt}}
% 
% \subsection{Attacks / Threats}
% 
% as bitsets per piece type
% number of attacks
% 
% \subsection{mas supongo?}
% 
% \subsection{Symmetry? / Relativity?}
% 
% \textbf{Motivation.}
% 
% BUCKETING
% 
% Medir el impacto de agregar simetría al fs. Red mas chica, inf mas rapida, mejor perf?
% 
% probar simetria, eventualmente probar con el mejor feature set de arriba, a ver si mejora poniendo a cada bloque individual simetria
% 
% \featureset{Half-Relative(H|V|HV)King-Piece}?
% 
% inspired by KP, build features relative to the position of the $\king$ King
% 
% \subsection{Statistical features?}
% 
% Define \featureset{k-All-All}
% 
% \featureset{King-All} is a subset of \featureset{All-All}.
% 
% Top P
% 
% Hacer un subset de \featureset{AA} (589824).
% 
% \begin{itemize}
% \item Destilar?
% \item Probar si es lo mismo quedarse con el TOP K de las mas comunes o con las que dice el performance.
% \item Catboost? PCA?
% \end{itemize}
