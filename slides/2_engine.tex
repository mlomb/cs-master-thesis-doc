
\section{Motor}

\begin{frame}
\frametitle{Motor de ajedrez}
Para evaluar las redes NNUEs es necesario un motor de ajedrez. \pause \\
\vspace{1em}
Buscamos construir un \textbf{motor de ajedrez clásico}, con \textbf{optimizaciones clásicas} pero \textbf{que use NNUEs} para evaluar posiciones.
\end{frame}

\begin{frame}
\frametitle{Minimax}
\textbf{Primera idea}: evalúo todas las posiciones a las que me puedo mover y elijo la mejor. \\
\vspace{1em}
\pause
Pero si extendemos la idea recursivamente... es el algoritmo \textbf{minimax}.
\begin{itemize}
\item \textcolor{ForestGreen}{$\triangle$} \textbf{Maximizing nodes}: nuestro jugador. Elige el movimiento que maximice la evaluación.
\item \textcolor{red}{\raisebox{0.1em}{\rotatebox[origin = c]{180}{$\triangle$}}} \textbf{Minimizing nodes}: el oponente. Elige el movimiento que minimiza la evaluación.
\end{itemize}
\end{frame}

\begin{frame}[shrink=7]
\frametitle{Minimax}
\begin{figure}
\centering
% https://tikz.dev/tikz-trees
\begin{tikzpicture}[level distance=18mm,level/.style={sibling distance=70mm/####1},line width=0.5mm,minimum size=1.5cm, inner sep=-10mm]
    % regular polygon, regular polygon sides=3
\tikzstyle{max node}=[regular polygon, regular polygon sides=3,draw=ForestGreen]
\tikzstyle{min node}=[regular polygon, regular polygon sides=3,shape border rotate=180,draw=red]
\node[max node]{-7}
child{
    node[min node]{-10} edge from parent[draw=black]
    child {
        node[max node]{10} edge from parent[draw=black]
        child {
            node[min node]{10} edge from parent[draw=black]
            child {
                node[max node]{10} edge from parent[draw=black]
            }
            child {
                node[max node]{$+\infty$} edge from parent[draw=black]
            }
        }
        child {
            node[min node]{5} edge from parent[draw=black]
            child {
                node[max node]{5} edge from parent[draw=black]
            }
        }
    }
    child {
        node[max node]{-10} edge from parent[draw=black]
        child {
            node[min node]{-10} edge from parent[draw=black]
            child {
                node[max node]{-10} edge from parent[draw=black]
            }
        }
    }
}
child {
    node[min node]{-7} edge from parent[draw=blue]
    child {
        node[max node]{5} edge from parent[draw=black]
        child {
            node[min node]{5} edge from parent[draw=black]
            child {
                node[max node]{7} edge from parent[draw=black]
            }
            child {
                node[max node]{5} edge from parent[draw=black]
            }
        }
        child {
            node[min node]{$-\infty$} edge from parent[draw=black]
            child {
                node[max node]{$-\infty$} edge from parent[draw=black]
            }
        }
    }
    child {
        node[max node]{-7} edge from parent[draw=black]
        child {
            node[min node]{-7} edge from parent[draw=black]
            child {
                node[max node]{-7} edge from parent[draw=black]
            }
            child {
                node[max node]{-5} edge from parent[draw=black]
            }
        }
    }
};
\end{tikzpicture}
\caption{Un árbol minimax de 4 de profundidad. El \enquote{mejor} movimiento para el jugador maximizador es el que lleva a la evaluación más alta, macada en \textcolor{blue}{azul}.}
\end{figure}
\end{frame}

\begin{frame}
\frametitle{Iterative deepening}
No queremos hacer minimax a una profundidad fija, si no a un tiempo fijo (100 milisegundos). \\
\vspace{1em}
\pause
\textbf{Iterative deepening} es una técnica que consiste en hacer minimax a profundidades crecientes, hasta que se acabe el tiempo. \\
\vspace{1em}
\pause
Che pero no pierdo todo el cómputo que hice en la iteración anterior? \pause \textbf{Si, pero...} \\
\end{frame}

\begin{frame}
\frametitle{Optimizaciones}
\begin{itemize}
\item<1-> Poda Alpha-beta (\href{https://mlomb.dev/slides/mcts}{anim})
\item<2-> Reordenamiento de movimientos (peor caso Minimax)
\begin{itemize}
\item MVV/LVA (Most Valuable Victim/Least Valuable Attacker)
\item ↓
\end{itemize}
\item<3-> Tablas de transposición: un caché
\end{itemize}
\end{frame}
