
\begin{frame}
\frametitle{Pairwise axes: motivación}

\storechessboardstyle{smallvert}{
    tinyboard,
    maxfield=a8,
    showmover=false,
    hmargin=false,
    hlabel=false,
    boardfontsize=15pt,
}

\newcommand{\raiseby}{-11.5ex}

\begin{figure}
\centering

\begin{tabular}{ccc}

\raisebox{\raiseby}{\chessboard[
    style=smallvert,
    clearboard,
    addblack={Ra7},
    addwhite={na3,pa4},
]}
\raisebox{\raiseby}{\chessboard[
    style=smallvert,
    clearboard,
    addblack={Ra8},
    addwhite={na2,pa3},
]}
\raisebox{\raiseby}{\chessboard[
    style=smallvert,
    clearboard,
    addblack={Ra6},
    addwhite={na3,pa4},
]}
\raisebox{\raiseby}{\chessboard[
    style=smallvert,
    clearboard,
    addblack={Ra5},
    addwhite={na1,pa3},
]}
\raisebox{\raiseby}{\chessboard[
    style=smallvert,
    clearboard,
    addblack={Ra6},
    addwhite={na2,pa3},
]}

$\hdots$

&
$ \rightarrow$
&

\raisebox{-9.5ex}{\chessboard[
    blackfieldcolor=white,
    blackfieldmaskcolor=white,
    maxfield=a8,
    style=smallvert,
    vlabel=false,
    border=false,
    trim=false,
    opacity=0.6,
    clearboard,
    addblack={Ra6},
    addwhite={na2,pa4},
    %
    color=red,
    shortenend=1.88ex,shortenstart=1.88ex, % espacio
    padding=-1ex,
    markstyle=leftborder,
    linewidth=0.4ex,
    markregion=a4-a6,
    linewidth=1.6ex,
    pgfstyle=circle,
    markfields={a4,a6},
    %
    color=blue,
    shortenend=1.88ex,shortenstart=1.88ex, % espacio
    padding=-1ex,
    markstyle=leftborder,
    linewidth=0.4ex,
    markregion=a2-a4,
    linewidth=1.6ex,
    pgfstyle=circle,
    markfields={a2,a4},
    %
]}

\\

\makecell{Configuraciones distintas,\\situaciones similares} &  & \makecell{Las mismas dos features\\(par rojo y par azul)}

\end{tabular}
\end{figure}
\end{frame}

\begin{frame}
\frametitle{Pairwise axes: motivación}
Comparando con el experimento anterior, es más específico en vez de más general:
\begin{center}
\enquote{\textit{there is a $\white$ White $\rook$ Rook in the 4th rank}} \\
vs. \\
\enquote{\textit{there is a $\black$ Black $\rook$ Rook next to a $\white$ White $\sympawn$ Pawn in the \enquote{a} file}}
\end{center}
\end{frame}

\begin{frame}
\frametitle{Pairwise axes: experimento}
\begin{table}
\small
\centering
\begin{adjustbox}{max width=\textwidth}
\begin{tabular}{ccccc}
\toprule
\bf D. & \bf \makecell{Block\\name} & \bf Definition & \bf \makecell{Num. of\\features} \\
\toprule
\depiction{PH} & PH & \makecell{
\vspace{0.2cm}
$(\featureset{Ranks} \times (\featureset{Roles} \times \featureset{Colors}) \times (\featureset{Roles} \times \featureset{Colors}))_{P}$ \\
P($\langle r, r_1, c_1, r_2, c_2 \rangle$): there is a piece in rank $r$ with role $r_1$\\ and color $c_1$ to the left of a piece with role $r_2$ and color $c_2$
} & 1152 \\
\toprule
\depiction{PV} & PV & \makecell{
\vspace{0.2cm}
$(\featureset{Files} \times (\featureset{Roles} \times \featureset{Colors}) \times (\featureset{Roles} \times \featureset{Colors}))_Q$ \\
Q($\langle f, r_1, c_1, r_2, c_2 \rangle$): there is a piece in file $f$ with role $r_1$\\ and color $c_1$ below a piece with role $r_2$ and color $c_2$
} & 1152 \\
\bottomrule
\end{tabular}
\end{adjustbox}
\end{table}
\end{frame}

\begin{frame}
\frametitle{Pairwise axes: experimento}
\begin{figure}[h]
\centering
\begin{adjustbox}{max width=\textwidth}
\begin{tabular}{cc}

\raisebox{-7ex}{\chessboard[
    setfen=2r4k/p5p1/Kpqp3p/8/1PP2Q2/P2P1RP1/8/8 b - - 12 45,
    showmover=false,
    opacity=0.6,
    %
    % TEMPLATE HORIZONTAL
    %color=red,
    %shortenend=1.88ex,shortenstart=1.88ex, % espacio
    %padding=-1ex,
    %markstyle=topborder,
    %linewidth=0.4ex,
    %markregion=d6-d3,
    %linewidth=1.6ex,
    %pgfstyle=circle,
    %markfields={d6,d3},
    %
    color=red,
    shortenend=1.88ex,shortenstart=1.88ex, % espacio
    padding=-1ex,
    markstyle=topborder,
    linewidth=0.4ex,
    markregion=a3-d3,
    linewidth=1.6ex,
    pgfstyle=circle,
    markfields={a3,d3},
    %
    color=red,
    shortenend=1.88ex,shortenstart=1.88ex, % espacio
    padding=-1ex,
    markstyle=topborder,
    linewidth=0.4ex,
    markregion=f3-g3,
    linewidth=1.6ex,
    pgfstyle=circle,
    markfields={f3,g3},
    %
    color=blue,
    shortenend=1.88ex,shortenstart=1.88ex, % espacio
    padding=-1ex,
    markstyle=topborder,
    linewidth=0.4ex,
    markregion=d3-f3,
    linewidth=1.6ex,
    pgfstyle=circle,
    markfields={d3,f3},
    %
    color=red,
    shortenend=1.88ex,shortenstart=1.88ex, % espacio
    padding=-1ex,
    markstyle=topborder,
    linewidth=0.4ex,
    markregion=b4-c4,
    linewidth=1.6ex,
    pgfstyle=circle,
    markfields={b4,c4},
    %
    color=blue,
    shortenend=1.88ex,shortenstart=1.88ex, % espacio
    padding=-1ex,
    markstyle=topborder,
    linewidth=0.4ex,
    markregion=c4-f4,
    linewidth=1.6ex,
    pgfstyle=circle,
    markfields={c4,f4},
    %
    color=red,
    shortenend=1.88ex,shortenstart=1.88ex, % espacio
    padding=-1ex,
    markstyle=topborder,
    linewidth=0.4ex,
    markregion=a6-b6,
    linewidth=1.6ex,
    pgfstyle=circle,
    markfields={a6,b6},
    %
    color=red,
    shortenend=1.88ex,shortenstart=1.88ex, % espacio
    padding=-1ex,
    markstyle=topborder,
    linewidth=0.4ex,
    markregion=c6-d6,
    linewidth=1.6ex,
    pgfstyle=circle,
    markfields={c6,d6},
    %
    color=blue,
    shortenend=1.88ex,shortenstart=1.88ex, % espacio
    padding=-1ex,
    markstyle=topborder,
    linewidth=0.4ex,
    markregion=b6-c6,
    linewidth=1.6ex,
    pgfstyle=circle,
    markfields={b6,c6},
    %
    color=blue,
    shortenend=1.88ex,shortenstart=1.88ex, % espacio
    padding=-1ex,
    markstyle=topborder,
    linewidth=0.4ex,
    markregion=d6-h6,
    linewidth=1.6ex,
    pgfstyle=circle,
    markfields={d6,h6},
    %
    color=red,
    shortenend=1.88ex,shortenstart=1.88ex, % espacio
    padding=-1ex,
    markstyle=topborder,
    linewidth=0.4ex,
    markregion=a7-g7,
    linewidth=1.6ex,
    pgfstyle=circle,
    markfields={a7,g7},
    %
    color=blue,
    shortenend=1.88ex,shortenstart=1.88ex, % espacio
    padding=-1ex,
    markstyle=topborder,
    linewidth=0.4ex,
    markregion=c8-h8,
    linewidth=1.6ex,
    pgfstyle=circle,
    markfields={c8,h8},
]}

&

\raisebox{-7ex}{\chessboard[
    setfen=2r4k/p5p1/Kpqp3p/8/1PP2Q2/P2P1RP1/8/8 b - - 12 45,
    showmover=false,
    opacity=0.6,
    %
    % TEMPLATE VERTICAL
    %color=red,
    %shortenend=1.88ex,shortenstart=1.88ex, % espacio
    %padding=-1ex,
    %markstyle=leftborder,
    %linewidth=0.4ex,
    %markregion=d6-d3,
    %linewidth=1.6ex,
    %pgfstyle=circle,
    %markfields={d6,d3},
    %
    color=red,
    shortenend=1.88ex,shortenstart=1.88ex, % espacio
    padding=-1ex,
    markstyle=leftborder,
    linewidth=0.4ex,
    markregion=a3-a6,
    linewidth=1.6ex,
    pgfstyle=circle,
    markfields={a3,a6},
    %
    color=blue,
    shortenend=1.88ex,shortenstart=1.88ex, % espacio
    padding=-1ex,
    markstyle=leftborder,
    linewidth=0.4ex,
    markregion=a6-a7,
    linewidth=1.6ex,
    pgfstyle=circle,
    markfields={a6,a7},
    %
    color=red,
    shortenend=1.88ex,shortenstart=1.88ex, % espacio
    padding=-1ex,
    markstyle=leftborder,
    linewidth=0.4ex,
    markregion=b4-b6,
    linewidth=1.6ex,
    pgfstyle=circle,
    markfields={b4,b6},
    %
    color=red,
    shortenend=1.88ex,shortenstart=1.88ex, % espacio
    padding=-1ex,
    markstyle=leftborder,
    linewidth=0.4ex,
    markregion=c4-c6,
    linewidth=1.6ex,
    pgfstyle=circle,
    markfields={c4,c6},
    %
    color=blue,
    shortenend=1.88ex,shortenstart=1.88ex, % espacio
    padding=-1ex,
    markstyle=leftborder,
    linewidth=0.4ex,
    markregion=c6-c8,
    linewidth=1.6ex,
    pgfstyle=circle,
    markfields={c6,c8},
    %
    color=red,
    shortenend=1.88ex,shortenstart=1.88ex, % espacio
    padding=-1ex,
    markstyle=leftborder,
    linewidth=0.4ex,
    markregion=d3-d6,
    linewidth=1.6ex,
    pgfstyle=circle,
    markfields={d3,d6},
    %
    color=red,
    shortenend=1.88ex,shortenstart=1.88ex, % espacio
    padding=-1ex,
    markstyle=leftborder,
    linewidth=0.4ex,
    markregion=f3-f4,
    linewidth=1.6ex,
    pgfstyle=circle,
    markfields={f3,f4},
    %
    color=red,
    shortenend=1.88ex,shortenstart=1.88ex, % espacio
    padding=-1ex,
    markstyle=leftborder,
    linewidth=0.4ex,
    markregion=g3-g7,
    linewidth=1.6ex,
    pgfstyle=circle,
    markfields={g3,g7},
    %
    color=red,
    shortenend=1.88ex,shortenstart=1.88ex, % espacio
    padding=-1ex,
    markstyle=leftborder,
    linewidth=0.4ex,
    markregion=h6-h8,
    linewidth=1.6ex,
    pgfstyle=circle,
    markfields={h6,h8},
]}


\\

\makecell{\depiction{PH} Pairwise horizontal (\featureset{PH})} &
\makecell{\depiction{PV} Pairwise vertical (\featureset{PV})}

\end{tabular}
\end{adjustbox}
\end{figure}
\end{frame}


\begin{frame}
\frametitle{Pairwise axes: experimento}
Los feature sets a entrenar son:
\begin{itemize}
\item \featureset{All} $\oplus$ \featureset{PH} (1920 features)
\item \featureset{All} $\oplus$ \featureset{PV} (1920 features)
\item \featureset{All} $\oplus$ \featureset{PH} $\oplus$ \featureset{PV} (3072 features)
\end{itemize}
\end{frame}

\begin{frame}
\frametitle{Pairwise axes: resultados}
\begin{table}
\centering
\begin{tabular}{ccccc}
\toprule
\bf Feature set  & \bf \makecell{Number\\of features} & \makecell{\bf Val. loss\\\textit{min}} & \makecell{\bf Rating\\\textit{elo (rel. to \featureset{All})}} \\
\toprule
\featureset{All} (reference) & 768 & 0.003134 & \textbf{0.0} \\
\midrule
\featureset{All} $\oplus$ \depiction{PH} & 1920 & 0.003033 & -38.2 $\pm$ 4.8 \\
\midrule
\featureset{All} $\oplus$ \depiction{PV} & 1920 & 0.002946 & -8.4 $\pm$ 5.0 \\
\midrule
\featureset{All} $\oplus$ \depiction{PH} $\oplus$ \depiction{PV} & 3072 & \textbf{0.002868} & -37.6 $\pm$ 4.9 \\
\bottomrule
\end{tabular}
\end{table}
\begin{itemize}
    \item Reducir el número de pairs puede llevar a una mejora por sobre \featureset{All} (ej. $\sympawn$)
\end{itemize}
\end{frame}
